\chapter{Druhá kapitola s matematikou \texorpdfstring{$\int\! f(x)\,\mathrm{d}x$}{int f(x) dx} v~názvu}
%% matematicke symboly nelze do zalozek v PDF vlozit - proto slouzi prikaz \texorpdfstring{toto se vysazi}{toto se
%% vlozi do zalozek v PDF} - nektere symboly vlozit lze, viz kapitolu 50 v dokumentaci
%% http://mirrors.ctan.org/macros/latex/contrib/hyperref/hyperref.pdf
%% viz take http://orgmode.org/worg/org-symbols.html
%%

%%%%%%%%%%%%%%%%%%%%%%%%%%%%%%%%%%%%%%%%%%%%%%%%%%%%%%%%%%%%%%
%%%%%%%%% UKAZKA OPAKOVANI MATEMATICKYCH SYMBOLU %%%%%%%%%%%%%
% 
% ukázka opakování na řádkovém zlomu ukázka opakování na řádkovém zlomu zlom $c+ a+ b$
% 
% ukázka opakování na řádkovém zlomu ukázka opakování na řádkovém zlomu zl $c- a- b$
% 
% ukázka opakování na řádkovém zlomu ukázka opakování na řádkovém zlomu zlo $c\cdot a\cdot b$
% 
% ukázka opakování na řádkovém zlomu ukázka opakování na řádkovém zlomu zl $c\setminus a\setminus b$
% 

Text text  text  text  text  text  text  text  text  text  text  text  text  text  text  text  text  text  text  text 
text  text  text  text  text  text  text  text  text  text  text  text  text  text  text  text  text  text  text  text 
text  text  text  text  text  text  text  text  text  text.

\begin{equation*}\label{E}\tag{rovnice}
 \int\! f(x)\,\mathrm{d}x
\end{equation*}
odkaz na rovnici s \uv{tagem} \eqref{E} -- pokud nechcete vzorci přidělit číslo, ale nějaký vlastní symbol, používejte
\uv{hvězdičkovaná} prostředí, tj. např. \verb+equation*+
\begin{equation}\label{EE}
 \iint\! f(x)\,\mathrm{d}x
\end{equation}
odkaz na druhy vzorec \eqref{EE}

\lipsum[1]
%%%%%%%%%%%%%%%%%%%%%%%%%%%%%%%%%%%%
%%%%%%%%% GENERUJ TEXT %%%%%%%%%%%%%

\shorthandoff{-} 
\lipsum[9-14]
\index{nějaká!položka}
\lipsum[35-41]

\section{Podkapitola}

\lipsum[50-53]
\index{jiná položka}
\shorthandon{-} 
%%%%%%%%%%%%%%%%%%%%%%%%%%%%%%%%%%%%

