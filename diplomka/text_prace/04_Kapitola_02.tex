\chapter{Konvergence trajektorií}
Jak již bylo naznačeno v úvodní kapitole, v některých případech používáme ke
zlepšení energie aktuální trajektorie backtracking. Nicméně implementovat
backtracking trajektorii až úplně zpět k ústí tunelu by bylo velice nákladné
a neefektivní, proto chceme nějakým způsobem backtracking napojit na již
spočítanou trajektorii, čímž si často můžeme ušetřit značné množství práce.
Uvažme tedy, že máme pozici backtracking trajektorie $ \tilde{\lambda}^i $
na řezu $ \theta_i $ a pozici dopředné trajektorie $ \lambda^i $ na tomtéž řezu.
Nejjednodušším přístupem v takové situaci je zkontrolovat, zda platí
$ \tilde{\lambda}^i \in \Delta \lambda^i $. Tuto kontrolu můžeme provádět v každé
iteraci backtrackingu neboť při ní stačí lineárně projít všechny atomy ligandu
a zkontrolovat jejich vzdálenosti oproti ligandu dopředné trajektorie. Ligandy
navíc typicky mívají velmi malý počet atomů, takže se skutečně jedná o nenáročný
výpočet. Tento přístup bude fungovat ve chvíli, kdy molekula ligandu po opuštění
úzkého hrdla zkonverguje do původní konformace, kterou zaujímá ligand v dopředné
trajektorii, k čemuž dojde zejména v situaci, kdy existuje lokální eneretické
minimum, do kterého se z backtracking konformace můžeme dostat gradientním sestupem.
Této situaci budeme říkat \textit{jednoduchá konvergence}.

\begin{figure}[ht]
\centering
\includegraphics[width=.5\hsize]{img/backtracking_simple.png}
\caption{Jednoduchá konvergence. Dopředná trajektorie je vyznačena tečkovaně,
backtracking plnou čárou.
}
\label{fig:simple_convergence}
\end{figure}


Příklad takové situace můžeme vidět na obrázku č. \ref{fig:simple_convergence}.
Při dopředném pohybu jsme uvázli v úzkém hrdle, ze kterého jsme s vhodnější
konformací spustili backtracking, který po několika iteracích na disku č. 4
samovolně zkonvergoval do pozice, téměř identické s pozicí dopředné trajektorie.
V tuto chvíli by byl backtracking ukončen a dále bychom pokračovali z disku č. 7.
Poznamenejme, že uvedený obrázek je čistě schématický a konvergenční proces
by v praxi zabral více kroků kvůli požadavkům na spojitost.

Jednoduchá konvergence může dobu výpočtu zkrátit, nicméně v praxi se často stává,
že ligand při backtrackingu zkonverguje do konformace, která je velmi podobná
konformaci dopředné trajektorie, ale liší se například rotací. Obě pozice tak
mohou mít velice podobnou energii, ale z mělkého lokálního minima se autonomě
nedostanou a backtracking by mohl dospět až k ústí tunelu.

\begin{figure}[ht]
\centering
\includegraphics[width=.5\hsize]{img/backtracking_bad.png}
\caption{Obrázek zachycuje situaci, ve které při backtrackingu dojde k opačné
rotaci ligandu, která v důsledku znemožní aplikaci jednoduché konvergence.
Dopředná trajektorie je vyznačena tečkovaně, backtracking plnou čárou.
}
\label{fig:backtracking_bad}
\end{figure}

Schéma č. \ref{fig:backtracking_bad} je ukázkou takové konstelace, ve které
oproti předchozímu příkladu při backtrackingu došlo k opačné rotaci ligandu
a algoritmus se tím dostal do jiného lokálního energetického minima. Obě situace
mouhou nastat se srovnatelnou pravděpodobností, neboť dokovací algoritmus je
náhodnostní a můžeme z něj ve stejné situaci na disku č. 6 dostat různý iniciální
krok backtrackingu.

Aby náš algoritmus v situaci, ke které dojde na řezu č. 4, mohl využít již spočítanou
dopřednou trajektorii, musíme zjistit, zda mezi pozicemi ligandu z backtracking trajektorie
$ \tilde{\lambda}^4 $ a dopředné trajektorie $ \lambda^4 $ neexistuje nějaká silná
energetická bariéra, která by v reálném systému činila pravděpodobnost přechodu
mezi těmito pozicemi nulovou. To podobně, jako u v úvodu zmíněné optimalizace
energie ligandu, budeme realizovat vypočtením dílčí spojité trajektorie.
V tomto případě ale půjde o spojitou trajektorii mezi pozicí $ \lambda^i $ dopředné
trajektorie a pozicí $ \tilde{\lambda}^i $ z backtracking algoritmu. Takový
výpočet je o něco obtížnější, neboť vedle aplikace omezujícího vzoru (kvůli
požadavku na spojitost) musíme ještě nějakým způsobem zajistit, abychom v průběhu
výpočtu atomy výchozího ligandu $ \tilde{a}_c \in \tilde{\lambda}^i $ tlačili
směrem ke korespondujícím atomům cílového ligandu $ a_c \in \lambda^i $.
Tento proces budeme nazývat aplikace \textit{slabého omezujícího vzoru} (nebo
zkráceně \textit{slabého vzoru}) a jeho implementaci si nastíníme v následující
sekci, na kterou navážeme podkapitolou věnovanou samotnému konvergenčnímu algoritmu.




\section{Slabý omezující vzor} \label{subsec:weak_pattern}
Implementace omezujících vzorů, tak aby fungovaly v souladu s našimi požadavky,
je poměrně komplikovaná a proto zde popíšeme jen ty části, které jsou nezbytné
pro pochopení konvergenčního algoritmu. Omezující vzory v principu fungují tak,
že atomům ligandu, které jsou mimo pozici požadovanou vzorem, přiřazují
penalizační vektor, jehož velikost roste se vzdáleností od ideální pozice
a je orientován ve směru od současné $ \tilde{a}_c $ k ideální pozici daného
atomu $ \tilde{a}_c $. Přesněji řečeno penalizační vektor $ \vec{p_c} $ pro
atom $ c $ definujeme jako
\begin{align*}
    \vec{p_c} = \vec{v} \cdot \max\left\{ 0, 1 - \frac{\delta}{\norm{\vec{v}}} \right\},
\end{align*}
kde $ \vec{v} $ je směrový vektor daný korespondujícími atomy, tedy
$ \vec{v} = a_c - \tilde{a}_c $, a $ \delta $ je parametr spojitosti z úvodní
kapitoly. Jak vidno, vektor je buďto nulový, pokud je aktuální pozice atomu vzdálena
o méně nebo rovno $ \delta $, nebo do něj v opačném případě přiřadíme vektor
realizující rozdíl mezi atomy ponížený o $ \delta $.

Vektory $ p_c $ můžeme chápat jakožto evaluaci silového pole daného omezujícím
vzorem. Suma $ E = \sum \norm{p_c} $ reprezentuje potenciální energii
ligandu v dané pozici vzhledem k silovému poli omezujícího vzoru. Energii $ E $
a vektory $ p_c $ pak použijeme v dokovacím algoritmu jako informace o jednom
ze silových polí působících na ligand. Tento algoritmus se snaží minimalizovat sumu
potenciálních energií přes všechna silová pole a vektory $ p_c $ využívá pro
směřování výpočtu.

V případě silného omezujícího vzoru, který zajišťuje spojitost, bereme za
ideální pozici atomů tu výchozí, čímž budeme silně penalizovat atomy, které se
z původní pozice vzdálí o více než $ \delta $. Slabý omezující vzor bude naopak
atomy z výchozí pozice přitahovat k atomům backtracking trajektorie. Tato konfigurace
potřebuje ještě jednu drobnou úpravu, aby její funkcionalita odpovídala našemu
původnímu záměru, což byla implementace spojitého přesunu od dopředné trajektorie
k backtracking trajektorii. Chceme totiž primárně respektovat silný omezující
vzor, aby ligandy atomu nemohly přeskočit nějakou energetickou bariéru, a posun
ligandu je až sekundární optimalizační cíl. Tento požadavek zajistíme tím, že
potenciální energii ligandu v silovém poli slabého vzoru budeme v dokovacím
algoritmu započítávat s nižší váhou.





\section{Konvergenční algoritmus} \label{subsec:convergence_algorithm}
S použitím slabého vzoru je samotný algoritmus pro výpočet konvergenční
trajektorie poměrně přímočará záležitost a zachycuje jej pseudokód č.
\ref{alg:convergence}. Algoritmus v každé iteraci nejprve na řádku
\ref{alg:convergence:docking} zavolá dokovací proceduru se silným omezujícím
vzorem $ \alpha(\lambda) $ a slabým vzorem $ \beta(\tilde{\lambda})$. Dokovací
procedura vrátí jednu nebo více nových konformací, přes které v dalším kroku
přeiterujeme a vybereme tu, jejíž průměrná vzdálenost od pozice $ \tilde{\lambda} $
je nižší než z aktuální pozice a zároveň se žádný atom neposunul o více než $ \delta $.
Pokud se nám nějakou konformaci podaří vybrat, prohlásíme ji za novou aktuální
konformaci $ \lambda $ a pokračujeme do další iterace. V opačném případě
vyhlásíme, že konvergence selhala a je potřeba ji spustit z jiné pozice
backtracking trajektorie.


\begin{algorithm}
\caption{Výpočet konvergenční trajektorie}
\label{alg:convergence}
\begin{algorithmic}[1]

\Function{convergence}{$ \lambda, \tilde{\lambda} $}
    \State $ d \gets \operatorname{AvgDst}(\lambda, \tilde{\lambda}) $
    \While {$\lambda \not\in \Delta\tilde{\lambda}$}
        \State $ L \gets $ Conformations found by docking algorithm using $ \alpha(\lambda) $
            and $ \beta(\tilde{\lambda})$ \label{alg:convergence:docking}
        \State $ d' \gets d $
        \For{$ \lambda' \in L $}
            \If {$ E(\lambda') \leq E_{max} $}
                \If {$ \lambda' \in \Delta\lambda
                        \wedge \operatorname{AvgDst}(\lambda', \tilde{\lambda}) < d $}
                    \State $ \lambda \gets \lambda' $
                    \State $ d \gets \operatorname{AvgDst}(\lambda, \tilde{\lambda}) $
                \EndIf
            \EndIf
        \EndFor
        \If {$ d = d' $}
            \Return \textbf{Failed}
        \EndIf
    \EndWhile
    \Return Convergence trajectory
\EndFunction


\end{algorithmic}
\end{algorithm}
