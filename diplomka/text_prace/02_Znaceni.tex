\chapter*{Přehled použitého značení}
\addcontentsline{toc}{chapter}{Přehled použitého značení}

Pro snažší orientaci v textu zde čtenáři předkládáme přehled základního značení, které se v celé práci vyskytuje.
\begin{flushleft}
\begin{longtable}[l]{ll} %% [l] tabulka je zarovnana vlevo; [c] zarovnani na stred; [r] zarovnani v pravo
  $\Rbb$                    & množina všech reálných čísel \\[1mm]
  $\Zbb$                    & množina všech celých čísel \\[1mm]
  $\Nbb$                    & množina všech přirozených čísel\\[1mm]
  $ \langle \cdot, \cdot \rangle \colon \Rbb^n \times \Rbb^n \to \Rbb $ & skalární součin \\[1mm]
  $ \emptyset $            & prázdná množina \\[1mm]
  $ |H| $                  & kardinalita množiny $H$ \\[1mm]
  $ \norm{v} $             & norma vektoru $v$ \\[1mm]
  $ Swap(x, y) $           & výměna obsahu dvou proměnných \\[1mm]
  $ Pop(L) $               & odstranění posledního prvku ze seznamu $L$ \\[1mm]
  $ Append(L, x) $         & přidání prvku $x$ na konec seznamu $L$ \\[1mm]
\end{longtable}
\end{flushleft}
