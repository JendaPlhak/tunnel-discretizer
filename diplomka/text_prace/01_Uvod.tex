\chapter*{Úvod}
\addcontentsline{toc}{chapter}{Úvod}

Vědět jakým způsobem spolu budou interagovat různé dvojice protein-ligand je
velice důležité pro pochopení řady fundamentálních biochemických procesů, které
nalézají uplatnění v mnoha praktických odvětvích jako je například farmacie.
Ligand interaguje s proteinem v jeho \texttt{aktivním místě} (centru), kde dochází
k enzymové reakci. Příkladem může být inhibice aktivního místa proteinu, které
jinak umožňuje viru napadnout buňku. Simulace navázání ligandu (ligand vstoupí
do aktivního místa a vznikne stabilní vazba mezi ligandem a proteinem) a jeho
uvolnění (uvolnění ligandu ze stabilního komplexu) je užitečná v mnoha praktických
aplikacích, neboť nám umožňuje indetifikovat ligandy, které se budou navazovat
v daném proteinu nejrychleji, případně získané informace můžeme použít k úpravě
ligandu tak aby se navazoval rychleji nebo modifikovat samotný protein aby k
navázání docházelo snáz nebo hůř podle toho jaké chování je žádoucí. Aktivní
centrum proteinu je mnohdy ukryto hluboko uvnitř molekuly, takže aby se ligand
mohl navázat, musí nejprve projít tzv. tunelem, který vede z povrchu proteinu
k jeho aktivnímu centru. V takovém případě je potřeba analyzovat, zda je
pravděpodobné, aby se ligand do aktivního centra skze tunel dostal.

Chemické systémy, jakým je protein interagující s ligandem, se řídí druhým
termodynamickým zákonem a mají proto tendenci minimalizovat jejich
celkovou potenciální energii. Prakticky to znamená, že nejpravděpodobnější
\textit{konformace} molekul (tj. pozice jejich atomů) jsou ty, která mají minimální
potenciální energii. Molekulární systém se ale může nacházet také v nějakém
lokálním energetickém minimu a s určitou pravděpodobností mezi těmito lokálními
minimy může přecházet. Pravděpodobnost přechodu pak ovlivňuje teplota systému a
velikost energetické bariéry, která musí být při přechodu překonána, přičemž platí
že čím menší bariéra, tím vyšší pravděpodobnost přechodu. Tím pádem pokud známe
energetický profil průchodu ligandu tunelem proteinu, tak můžeme kvantifikovat
pravděpodobnost s jakou ligand tunelem projde. Funkci která pro zadanou konformaci
spočítá aproximaci její potenciální energie nazýváme \textit{silové pole}. Analýzou
potenciální energie, kterou získáme ze silového pole, pak můžeme vypočítat
s jakou pravděpodobností se bude daná konformace vyskytovat v reálném chemickém
systému. Díky tomu je možné porovnat dva chemické komplexy tvořené
dvojicemi protein-ligand (jakými jsou třeba dvě možné konformace ligandu
v aktivním centru jednoho proteinu) a rozhodnout, který komplex má vyšší
pravděpodobnost zformování v reálném systému.

Pokud chceme studovat navázání a uvolnění ligandu z aktivního centra, potřebujeme
být schopni vyhodnotit potenciální energii ligandu procházejícího tunelem
z povrchu proteinu do jeho aktivního centra a naopak. Ligand se v aktivním
jádře proteinu naváže pokud v tomto místě existuje silné energetické minimum
a v průběhu průchodu tunelem nenarazí na žádnou významnou energetickou bariéru
(gradient tunelu by měl být s menšími lokálními výkyvy stále klesající od
vstupu do tunelu až na jeho konec v aktivním místě). Pokud tunel obsahuje nějakou
silnou odpudivou bariéru, je pravděpodobné, že se ji ligandu nepodaří překonat
a tunelem neprojde. Poznamenejme, že energetický profil tunelu je pro každý
ligand unikátní a je tedy potřeba jej pro každou variaci ligandu vyhodnotit
znovu.

Navázání ligandu v aktivním jádře proteinu se obvykle počítá pomocí takzvaného
molekulárního dokingu. Algoritmus molekulárního dokingu prochází konformační
prostor molekulárního komplexu protein-ligand a snaží lokalizovat energetická
minima. Výstupem molekulárního dokingu je jeden nebo více protein-ligand
konformačních komplexů společně s jejich potenciální energií. Díky tomu
uživatel získá informaci o tom, který ligand se na protein váže s nižší potenciální
energií a může se na základě této informace pokoušet ligand nebo protein
dále upravovat. Problémem je, že algoritmus molekulárního dokingu vypočítá pouze
statické pozice ligandu a proteinu - nevygeneruje \textit{trajektorii} ligandu
(pohyb ligandu tunelem v čase), což znamená, že tento algoritmus sám o sobě
nestačí na to abychom mohli studovat přenos ligandu z povrchu proteinu
do jeho aktivního centra skrze jeho tunel.

V této práci popíšeme některé části nové metody pro výpočet potenciální energie
trajektorie ligandu, která umožňuje studovat proces navázání i uvolnění ligandu
včetně jeho cesty skrze tunel k aktivnímu centru. Naše metode je založena na
algirtmu molekulárního dokingu, který používá tak, že iterativně dokuje ligand
podél trajektorie tunelu a vyhodnocuje jeho potenciální energii. Náš doking
pracuje s hybridními silovými poli - jedná se o kombinaci chemického silového
pole, které se používá k výpočtu potenciální energie komplexu protein-ligand,
a omezujícího silového pole, které zmenšuje velikost prostoru potenciálních
konformací ligandu. Díky tomu je pozice ligandu v každém místě tunelu
omezena pouze na oblast definovaného prostoru. Skrze tunel zadaného proteinu
může existovat velmi mnoho cest, po kterých můžeme ligand provést. Prostor
všech potenciálních cest proto prohledáváme pomocí heuristického algoritmu
s backtrackingem. Naše metoda byla implementována uživatelsky přívětivém
nástroji CaverDock, který je navržen tak, aby byl schopen maximalizovat utilizaci
paralelních výpočetních architektur.

Tato práce se bude primárně soustředit na informaticko-matematickou stranu
daného problému: poskytneme obecný náhled na celý algoritmus a pak se budeme
soustředit zejména na dvě jeho části, za jejichž návrh a implementaci byl
zodpovědný autor této diplomové práce. Konkrétně se jedná o algoritmus pro
diskretizaci tunelu a algoritmus pro výpočet konvergence ligandu. Paralelně
s touto prací vzniká ještě článek popisující další části našeho algoritmu
do větších detailů společně s evaluací jeho celkového výkonu. Rovněž je v přípravě
publikace, která se zabývá biochemickými tématy, jakými jsou nastavení
vstupních parametrů algoritmu, interpretace výsledků nebo vyhodnocení výsledků
testování naší metody v praxi na mnoha dvojicích proteinů a ligandů.





\section{Obecný přehled}
V této sekci popíšeme základní koncept naší metody. Detailní popis zmíněných
dvou částí, které jsou stěžejními tématy této práce, bude následovat v dalších
dvou kapitolách. Naše metoda je založena na řízeném iterativním pohybu ligandu
skrze tunel, díky čemuž se vyhneme časově náročnému výpočtu stochastického
pohybu ligandu, který je typický pro simulace molekulární dynamiky.

Prvním krokem naší metody je diskretizace tunelu jejímž výstupem je posloupnost
omezujících podmínek, díky kterým můžeme definovat pozici ligandu v tunelu a
tím pádem i směr jeho pohyb tunelem - dopředu a dozadu. Na takto diskretizovaném
tunelu je poté ligand iterativně dokován na sérii po sobě jdoucích pozic v tunelu,
což nám umožňuje simulovat proces navázání ligandu nebo naopak jeho uvolnění.





\subsection{Diskretizace tunelu}
K tomu abychom mohli s ligandem tunelem iterativně procházet, potřebujeme nějakým
způsobem omezit prostor, ve kterém může být ligand umístěn.
Omezení se v našem případě realizuje pomocí posloupnosti koulí, které aproximují
geometrii tunelu. Takovouto aproximaci můžeme získat například pomocí nástroje
Caver \cite{Caver}. Na vstupu algoritmu tedy máme zmíněnou posloupnost
koulí, kterou náš algoritmus transformuje na posloupnost $ n $ řezů - kruhů
$ \theta_1, \dots, \theta_n $. Tyto řezy generujeme tak, abychom tunel rozdělili
na pláty, které mají shora omezenou tloušťku (to proto abychom nevynucovali
příliš velké změny v pozici ligandu mezi jednotlivými kroky výpočtu). Prakticky
se jedná o 2D kruhy, které omezují prostor, na který můžeme ligand zadokovat.
Cestu ligandu tunelem pak definujeme jako posloupnost umístění vybraného atomu
ligandu na po sobě jdoucí řezy (výběr atomu je libovolný, ale musí být stejný
pro všechny řezy). Popis algoritmu pro diskretizaci tunelu je obsahem první kapitoly.

\begin{figure}[t]
\centering
\includegraphics[width=.8\hsize]{img/tun.pdf}
\caption{Schématický 2D pohled na průchod tunelem. Vybraný atom ligandu je
postupně umisťován na po sobě jdoucí disky. Tím že nazajišťujeme spojitý pohyb
ligandu, můžeme vidět, že mezi řezy $\theta_6$ a $\theta_7$ dochází k prudké rotaci, v
důsledku čehož nedetekujeme úzké a potenciálně neprůchodné hrdlo tunelu.}
\label{fig:lower-bound}
\end{figure}





\subsection{Doking s omezujícími podmínkami}
Konformace ligandu $ \lambda $ je definována pozicí atomů
$ \lambda = {a}^m_{i = 1} $ v klasické kartézské souřadné soustavě. Maje
k dispozici diskretizaci tunelu, můžeme vybrat atom ligandu $ a_c \in \lambda $,
a poté iterativně provádět doking ligandu na disky $ \theta_1, \dots, \theta_n $,
přičemž budeme požadovat, aby výsledná konformace ligandu po dockingu na
$ i $-tý disk splňovala $ a_c \in \theta_i $. Tímto způsobem ligand přinutíme
k tomu, aby prošel celým tunelem.

Trajektorie ligandu vygenerovaná tímto způsobem navzorkuje profil tunelu bez
výraznějších mezer (to znamená, že ligand nebude schopen překonat (přeskočit) například
uzká hrdla tunelu), nicméně problémem je, že nemusí být spojitá (ligand se
může mezi sousedními řezy volně rotovat). Tuto nespojitou trajektorii budeme
používat jakožto dolní odhad na transportní energii. Příklad takové trajektorie
je vykreslen na obrázku č. \ref{fig:lower-bound}. Jak můžeme vidět, atom $ a_c $
je přichycen ke kruhu a jeho posunem skrze tunel modelujeme transportní
proces. Na obrázku lze také vidět příklad nespojitého pohybu - konkrétně
se jedná o otočení ligandu mezi řezy $ \theta_7$ a $\theta_8$.

Spojitou trajektorii můžeme vypočítat tak, že při přechodu mezi po sobě jdoucími
řezy omezíme pohyb každého z atomů na nějaké jeho $ \delta $-okolí. Přesněji řečeno
budeme požadovat, aby vzdálenost atomů v nové konformaci ligandu $ \lambda^{i + 1} $
od příslušných atomů v předchozí konformaci $ \lambda^{i} $ byla shora omezená
konstantou $ \delta $. Formálně
\begin{align}
    a_j \in \lambda^{i}, b_j \in \lambda^{i + 1} \Rightarrow |a_j - b_j| < \delta.
    \label{eq:pattern}
\end{align}
V této situaci řekneme, že $ \lambda^i $ je vzorem omezujícím konformaci
$ \lambda^{i+1} $ a budeme značit $ \lambda^{i+1} \in \Delta \lambda^i $ jestliže
je splněna podmínka (\ref{eq:pattern}).

\begin{figure}[t]
\centering
\includegraphics[width=.5\hsize]{img/tun-2.pdf}
\caption{Schématický 2D pohled na průchod tunelem, při kterém je předchozí pozice
ligandu použita jakožto omezující vzor. Omezení pohybu atomů zapříčiní, že geometrické
úzké hrdlo mezi řezy $\theta_6$ a $\theta_7$ je detekováno neboť ligand nyní
nemá možnost rotace. Ve výsledném energetickém profilu bychom pak viděli prudké
zvýšení energetického potenciálu v místě zadokování ligandu na řez č. 8.}
\label{fig:continuous_transition}
\end{figure}

V průběhu výpočtu pak kdykoliv počítáme další konformaci $ \lambda^{i+1} $,
omezíme pohyb ligandu na aktuální vzor $ \lambda^{i} $. Díky tomu budeme mít
garantováno, že $ \lambda^{i+1} \in \Delta \lambda^i $, z čehož plyne, že
transformace konformací ligandu mezi po sobě jdoucími řezy bude spojitá
(shora omezená konstantou $ \delta $). Na tomto místě musíme poznamenat, že aby
výše uvedený postup dával smysl, musí platit, že každé dva po sobě jdoucí řezy jsou
od sebe vzdáleny o vzdálenost menší než $ \delta $. Demonstraci použití
omezujícího vzoru můžeme vidět na obrázku č. \ref{fig:continuous_transition}. Jak
můžeme vidět, omezující vzor nedovolí ligandu rotaci jakou jsme mohli pozorovat
v přechozím případě (obrázek č. \ref{fig:lower-bound}), což nám umožní detekovat
úzké geometrické hrdlo.





\subsection{Hledání minimální spojité trajektorie}
Iterativním dokováním za použití výše popsaného omezení na konformační vzor můžeme
získat spojitou trajektorii. Našim cílem ovšem bude najít trajektorii, jejíž
celková energie bude co nejmenší. Z tohoto důvodu chceme ligandu
umožnit optimalizovat jeho pozici na každém z disků tak, aby jeho potenciální
energie byla co nejmenší. Toto minimum ale již po první iteraci nemusí být dostupné
kvůli omezením, které na atomy ligandu klade omezující vzor. Proto na každém
z řezů $ \theta_i $ budeme pro výchozí konformaci $ \lambda^i_j $ hledat dílčí
trajektorii (spojitou transformaci ligandu) $ \lambda^i_{j+1} \in \Delta \lambda^i_j,
\lambda^i_{j+2} \in \Delta \lambda^i_{j+1}, \dots $, která ve výsledku
zlepší potenciální energii ligandu oproti jeho výchozí konformaci $ \lambda^i_j $.
Tyto kroky budeme nazývat optimalizační kroky, neboť umožňují ligandu transformaci
do nízkoenrgetické konformace, která ihned po přechodu z $ \theta_{i - 1} $
na $ \theta_i $ nemusí být dosažitelná.

\begin{figure}[t]
\centering
\includegraphics[width=.8\hsize]{img/tun-3.pdf}
\caption{Srovnání průchodu ligandu tunelem, při kterém se zasekne (vlevo) a
průchodu s alternativní výchozí konformací, díky které se ligandu podaří
úzké hrdlo překonat (vpravo).
}
\label{fig:backtracking}
\end{figure}

Výše uvedený postup bude obecně preferovat pohyb ligandu, který vede ve směru
největšího energetického gradientu. Ačkoliv je takový scénář v reálné situaci
nejpravděpodobnější, může se také stát, že dojde k posunu ligandu do nějaké
jiné konformace, která mu může pomoci překonat energetickou bariéru, která se
vyskytuje v tunelu někde dále. Uvažme například situaci na obrázku
č. \ref{fig:lower-bound}. V závislosti na výchozí orientaci ligand úzké hrdlo
může, ale také nemusí překonat. Kvůli tomu musíme začít uvažovat vícero potenciálních
trajektorií.

Počet všech možných spojitích trajektorií je velmi vysoký - roste exponenciálně vzhledem
k počtu řezů: přechod na následující disk může změnit pozici ligandu, orientaci
nebo konformaci (relativní pozici atomů v rámci ligandu). Vyčerpávajcí zkoušení
všech možných trajektorií tedy nepřipadá v úvaru vzhledem k tomu, že čas
potřebný k zadokování ligandu s omezeními se obvykle pohybuje mezi stovkami
milisekund až sekundami. Kvůli tomu jsme byli nuceni zavést jednoduchou heuristiku:
pokud je potenciální energie pozice $ \lambda^i_j $ významně větší než potenciální
energie nějaké již známé konformace $ \lambda^i_{low} $ (například konformace
získaná během výpočtu dolního odhadu na energii trajektorie), nastavíme
$ \lambda^i_j = \lambda^i_{low} $ a začneme prohledávat tunel z tohoto místa
směrem dozadu přes řezy $ \theta_{i-1}, \theta_{i-2}, \dots $. Backtracking
skončí ve chvíli kdy dojde ke konvergenci dopředné a zpětné trajektorie a jsou
navázány nebo v případě že se dostaneme na úplný začátek tunelu.
Algoritmus pro detekci konvergence trajektorií je obsahem druhé kapitoly.

